\documentclass[11pt,letterpaper,sans]{moderncv}        % possible options include font size ('10pt', '11pt' and '12pt'), paper size ('a4paper', 'letterpaper', 'a5paper', 'legalpaper', 'executivepaper' and 'landscape') and font family ('sans' and 'roman')
\usepackage{xspace}
\newcommand\eat[1]{}
%\newcommand{\ie}{{\it i.e.,}\xspace}

% moderncv themes
%\moderncvstyle{classic}                             % style options are 'casual' (default), 'classic', 'oldstyle' and 'banking'
\moderncvstyle{classic}
\moderncvcolor{black}                               % color options 'blue' (default), 'orange', 'green', 'red', 'purple', 'grey' and 'black'
%\renewcommand{\familydefault}{\sfdefault}         % to set the default font; use '\sfdefault' for the default sans serif font, '\rmdefault' for the default roman one, or any tex font name
\renewcommand*{\subsectionfont}{\large\mdseries\upshape}
\renewcommand*{\subsectionstyle}[1]{\subsectionfont\textbf{#1}}
%\nopagenumbers{}                                  % uncomment to suppress automatic page numbering for CVs longer than one page

% character encoding
%\usepackage[utf8]{inputenc}                       % if you are not using xelatex ou lualatex, replace by the encoding you are using
%\usepackage{CJKutf8}                              % if you need to use CJK to typeset your resume in Chinese, Japanese or Korean

% adjust the page margins
\usepackage[scale=0.75]{geometry}
%\setlength{\hintscolumnwidth}{3cm}                % if you want to change the width of the column with the dates
%\setlength{\makecvtitlenamewidth}{10cm}           % for the 'classic' style, if you want to force the width allocated to your name and avoid line breaks. be careful though, the length is normally calculated to avoid any overlap with your personal info; use this at your own typographical risks...

% personal data
\name{Aurojit}{Panda}
\title{Curriculum Vitae}                               % optional, remove / comment the line if not wanted
\address{1176 University Ave, Apt 107}{Berkeley, CA 94702}   % optional, remove / comment the line if not wanted; the "postcode city" and "country" arguments can be omitted or provided empty
\phone[mobile]{+1~(401)~323~1524}                   % optional, remove / comment the line if not wanted; the optional "type" of the phone can be "mobile" (default), "fixed" or "fax"
%\phone[fixed]{+2~(345)~678~901}
%\phone[fax]{+3~(456)~789~012}
\email{apanda@cs.berkeley.edu}                               % optional, remove / comment the line if not wanted
\renewcommand*\httplink[2][]{{\urlstyle{sf}\expandafter\href#2}}
\homepage{{https://www.cs.berkeley.edu/~apanda/}{www.cs.berkeley.edu/{\textasciitilde}apanda}}                         % optional, remove / comment the line if not wanted
%\social[linkedin]{john.doe}                        % optional, remove / comment the line if not wanted
%\social[twitter]{jdoe}                             % optional, remove / comment the line if not wanted
%\social[github]{jdoe}                              % optional, remove / comment the line if not wanted
%\extrainfo{additional information}                 % optional, remove / comment the line if not wanted
%\photo[64pt][0.4pt]{picture}                       % optional, remove / comment the line if not wanted; '64pt' is the height the picture must be resized to, 0.4pt is the thickness of the frame around it (put it to 0pt for no frame) and 'picture' is the name of the picture file
%\quote{Some quote}                                 % optional, remove / comment the line if not wanted

% to show numerical labels in the bibliography (default is to show no labels); only useful if you make citations in your resume
%\makeatletter
%\renewcommand*{\bibliographyitemlabel}{\@biblabel{\arabic{enumiv}}}
\makeatother
%\renewcommand*{\bibliographyitemlabel}{[\arabic{enumiv}]}% CONSIDER REPLACING THE ABOVE BY THIS
\renewcommand*{\bibliographyitemlabel}{$\bullet$}% CONSIDER REPLACING THE ABOVE BY THIS

% bibliography with mutiple entries
\usepackage{multibib}
\newcites{conference,workshop,demos,trs}{{Conferences},{Workshops},{Demos},{Technical Reports}}
%----------------------------------------------------------------------------------
%            content
%----------------------------------------------------------------------------------
\begin{document}
%\begin{CJK*}{UTF8}{gbsn}                          % to typeset your resume in Chinese using CJK
%-----       resume       ---------------------------------------------------------
\makecvtitle
\section{Research Interests}
Distributed systems, networking

\section{Education}
\cventry{2011--present}{PhD student}{University of California}{Berkeley, CA}{}{Advisor: Scott Shenker}  % arguments 3 to 6 can be left empty
\cventry{2004--2008}{Sc.B. Math--Computer Science}{Brown University}{Providence, RI}{}{Honors in Math--Computer Science\\
Advisor: Meinolf Sellmann}

\section{Awards}
\cvlistitem{Best Student Paper, SIGCOMM 2015}
\cvlistitem{Best Paper, EuroSys 2013}
\cvlistitem{Qualcomm Innovation Fellowship 2012}

\section{Research}
\cvlistitem {\textbf{ZCSI}: Network operators are increasingly moving from hardware middleboxes to software network
functions. The move to software reduces vendor lock-in, makes it easier to deploy new features, and provides benefits
for consolidation. The current model for deploying these network functions envisions using virtual machines (or
containers) connected using virtual switches. While this model has served workloads in the cloud well, NFs need to
process millions of packets per-second while adding no more than a few microseconds of delay in order to meet SLAs. VMs
and containers add high-overheads for this. ZCSI is a new NF framework which leverages software isolation to provide the
same guarantees as VMs without incurring this overhead. Furthermore, ZCSI allows NFs to be written in high-level managed
languages, enhancing developer productivity and allowing faster development. ZCSI achieves this while improving
performance when compared to today's deployments.}

\cvlistitem {\textbf{Model Checking for Networks}: Networks include middleboxes and other stateful elements whose
    behavior is dictated by previously encountered packets. We developed a tool to verify isolation invariants (can
    packets with certain characteristics be delivered to certain end hosts, and can data originating at a particular
    host end up somewhere). We also developed theoretical foundations showing that if one uses middlebox models with
    certain simple limitations and runs checks in loop free network then verification is guaranteed to terminate.}

\cvlistitem {\textbf{Data-Aware Cluster Scheduling for Sampling Jobs}: An increasing portion of cluster computing jobs
use sampled data, either to reduce computational time using approximate querying or when computing stochastic machine
learning functions. In this work, we looked at how cluster schedulers could use sampling information to significantly
improve the performance of cluster computing jobs.}

\cvlistitem {\textbf{Network Consistency Models}: Networks increasingly enforce policy, which ultimately requires that
several of the middleboxes/routers in a network have a consistent view of the network. However, the CAP theorem says
that this must necessarily come at the cost of giving up availability in the presence of partitions. In this work we
analyzed the effects of this trade-off.}

\cvlistitem {\textbf{Data Driven Connectivity}: Minimizing the network disruption due to link or router/switch failures
    continues to be an important concern.
    We developed an online algorithm that provides perfect forwarding
resilience \textit{i.e.,} guarantees that in an uncongested, connected network no set of link failures results in a packet drop. We
have also had subsequent work on theory that shows the limits of resilience that can be achieved using just static
routing tables with no updates.}

\cvlistitem {\textbf{SDNv2}: The rise of NFV and other similar technologies has meant that general purpose computers are
increasingly available at the network edge. We designed a new internet architecture that allowed ISPs to leverage these
computers to provide increased flexibility to the customers.}

\cvlistitem {\textbf{SMIRC}: We asked the question, can secure multiparty computation (SMPC) can be used to compute BGP
routing information. The use of SMPC would allows BGP computation to be centralized without requiring ASes to reveal
private policy information, speeding up BGP convergence without requiring ASes to change their trust model.}

\cvlistitem {\textbf{Recursive SDN}: We looked at how one could scale SDN controllers so they can be used to control ISP
    networks which tend to both be numerically larger, and geographically wide-spread. Our solution is to use hierarchy
    and recursive algorithms.}


    \cvlistitem {\textbf{BlinkDB}: We built a fast, approximate database on top of Spark. BlinkDB used sampling and
    innovative statistical techniques to provide approximate answers and quality measurements for arbitrary SQL query.
BlinkDB has since been deployed at companies like Facebook, is commercially supported by other companies.}

\cvlistitem {\textbf{Symmetry Breaking for Constraint Programming}: We looked at methods by which one could speed up
solving constraint satisfaction problems. These problems are widely used (for instance to schedule UPS delivery trucks
and NFL games) and are susceptible to an exponential growth in time, particularly when many equivalent (symmetric)
solutions exist. }

\section{Teaching Experience}
\cventry{\hspace{2em}\ \ Fall '13\newline Fall '12}{TA for EE122 (Undergraduate Networking)}{UC Berkeley}{Berkeley, CA}{}{
\begin{itemize}
\item Taught about 20 students in a weekly discussion section.
\item Designed problem sets and projects, ran a project involving plug computers
\end{itemize}}
\cventry{2006--2008}{Undergraduate TA}{CS Department, Brown University}{Providence, RI}{}{
\begin{itemize}
\item TAed CS51 (Models of Computation), CS138 (Distributed Computing), CS166 (Security), CS167 (Operating Systems) and CS169 (Operating Systems Lab).
\item Helped with designing new content and projects for distributed systems and security.
\item Helped design and grade problem sets and projects.
\end{itemize}
}
\cventry{Fall '05}{Reader}{Math Department, Brown University}{Providence, RI}{}{
    \begin{itemize}
    \item Graded problem sets and exams for Honors Linear Algebra.
    \end{itemize}
}
\section{Industry Experience}
\cventry{2008--2011}{Software Developer}{Microsoft}{Redmond, WA}{}{\begin{itemize}
\item Worked on the kernel for Midori, an experimental operating system based on Singularity.
\item Designed, implemented and maintained Midori's performance counting and event logging infrastructure.
\item Designed and implemented timer infrastructure for Midori. Midori relied purely on core-local time and allowed applications to specify a tradeoff between
time accuracy and performance.
\item Ported Midori from x86-64 to NVidia's Tegra2 multicore ARM processor.
\end{itemize}}%
\cventry{Summer '07}{Software Engineering Intern}{Electronic Arts}{Redwood City, CA}{}{\begin{itemize}
\item Developed memory management tools for the PlayStation 3.
\end{itemize}}
\cventry{Summer '06}{Software Engineering Intern}{Bloomberg LP}{New York, NY}{}{\begin{itemize}
\item Developed network debugging tools for discovering the cause for latency spikes in data transfer from financial markets.
\end{itemize}}

\section{Publications}
\nociteconference{*}
\bibliographystyleconference{plainyr-rev}
\bibliographyconference{conference}                   % 'publications' is the name of a BibTeX file

\nociteworkshop{*}
\bibliographystyleworkshop{plainyr-rev}
\bibliographyworkshop{workshop}                   % 'publications' is the name of a BibTeX file


\nocitedemos{*}
\bibliographystyledemos{plainyr-rev}
\bibliographydemos{demos}                   % 'publications' is the name of a BibTeX file

\nocitetrs{*}
\bibliographystyletrs{plainyr-rev}
\bibliographytrs{tr}                   % 'publications' is the name of a BibTeX file
\section{References}
Available on request
\end{document}


%% end of file `template.tex'.
