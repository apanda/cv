%% start of file `template.tex'.
%% Copyright 2006-2013 Xavier Danaux (xdanaux@gmail.com).
%
% This work may be distributed and/or modified under the
% conditions of the LaTeX Project Public License version 1.3c,
% available at http://www.latex-project.org/lppl/.


\documentclass[11pt,letterpaper,sans]{moderncv}        % possible options include font size ('10pt', '11pt' and '12pt'), paper size ('a4paper', 'letterpaper', 'a5paper', 'legalpaper', 'executivepaper' and 'landscape') and font family ('sans' and 'roman')
\newcommand\eat[1]{}

% moderncv themes
\moderncvstyle{classic}                             % style options are 'casual' (default), 'classic', 'oldstyle' and 'banking'
\moderncvcolor{blue}                               % color options 'blue' (default), 'orange', 'green', 'red', 'purple', 'grey' and 'black'
%\renewcommand{\familydefault}{\sfdefault}         % to set the default font; use '\sfdefault' for the default sans serif font, '\rmdefault' for the default roman one, or any tex font name
%\nopagenumbers{}                                  % uncomment to suppress automatic page numbering for CVs longer than one page

% character encoding
%\usepackage[utf8]{inputenc}                       % if you are not using xelatex ou lualatex, replace by the encoding you are using
%\usepackage{CJKutf8}                              % if you need to use CJK to typeset your resume in Chinese, Japanese or Korean

% adjust the page margins
\usepackage[scale=0.75]{geometry}
%\setlength{\hintscolumnwidth}{3cm}                % if you want to change the width of the column with the dates
%\setlength{\makecvtitlenamewidth}{10cm}           % for the 'classic' style, if you want to force the width allocated to your name and avoid line breaks. be careful though, the length is normally calculated to avoid any overlap with your personal info; use this at your own typographical risks...

% personal data
\name{Aurojit}{Panda}
\title{Curriculum Vitae}                               % optional, remove / comment the line if not wanted
\address{1176 Universty Ave Apt 107}{Berkeley, CA 94702}   % optional, remove / comment the line if not wanted; the "postcode city" and "country" arguments can be omitted or provided empty
\phone[mobile]{+1~(401)~323~1524}                   % optional, remove / comment the line if not wanted; the optional "type" of the phone can be "mobile" (default), "fixed" or "fax"
%\phone[fixed]{+2~(345)~678~901}
%\phone[fax]{+3~(456)~789~012}
\email{apanda@cs.berkeley.edu}                               % optional, remove / comment the line if not wanted
\homepage{www.cs.berkeley.edu/~apanda/}                         % optional, remove / comment the line if not wanted
%\social[linkedin]{john.doe}                        % optional, remove / comment the line if not wanted
%\social[twitter]{jdoe}                             % optional, remove / comment the line if not wanted
%\social[github]{jdoe}                              % optional, remove / comment the line if not wanted
%\extrainfo{additional information}                 % optional, remove / comment the line if not wanted
%\photo[64pt][0.4pt]{picture}                       % optional, remove / comment the line if not wanted; '64pt' is the height the picture must be resized to, 0.4pt is the thickness of the frame around it (put it to 0pt for no frame) and 'picture' is the name of the picture file
%\quote{Some quote}                                 % optional, remove / comment the line if not wanted

% to show numerical labels in the bibliography (default is to show no labels); only useful if you make citations in your resume
%\makeatletter
%\renewcommand*{\bibliographyitemlabel}{\@biblabel{\arabic{enumiv}}}
\makeatother
\renewcommand*{\bibliographyitemlabel}{[\arabic{enumiv}]}% CONSIDER REPLACING THE ABOVE BY THIS

% bibliography with mutiple entries
\usepackage{multibib}
\newcites{conference,workshop,demos}{{Conferences},{Workshops},{Demos}}
%----------------------------------------------------------------------------------
%            content
%----------------------------------------------------------------------------------
\begin{document}
%\begin{CJK*}{UTF8}{gbsn}                          % to typeset your resume in Chinese using CJK
%-----       resume       ---------------------------------------------------------
\makecvtitle
\section{Research Interests}
Distributed Systems, Networking, Formal Verification, Network Resilience, Software-Defined Networking.

\section{Education}
\cventry{2011--present}{PhD student}{University of California}{Berkeley, CA}{}{Advisor: Scott Shenker}  % arguments 3 to 6 can be left empty
\cventry{2004--2008}{Sc.B. Math-Computer Science}{Brown University}{Providence, RI}{}{Honors in Math-Computer Science\\
Advisor: Meinolf Sellmann}

\section{Awards}
\cvlistitem{Best Paper, EuroSys 2012}
\cvlistitem{Qualcomm Innovation Fellowship 2012}

\section{Research}
\cvlistitem {Model Checking for Networks, a project to build tools to check if network policies are correctly enforced given a network topology and policy state for individual
elements.}
\eat{Network policy (for instance isolation policies implemented by firewalls or caching as done by web proxies) enforcement requires
a variety of middleboxes to act correctly in concert: for instance isolation policies require not just that firewalls be correctly configured but also require
that no previous middleboxes strip information required by the firewall. This project focuses on providing tools that given network topology and higher level
policies produce proofs that the policies are enforced or counterexamples where they are not.}

\cvlistitem {Network Consistency Models, applies the CAP theorem to SDN networks, and the minimal consistency model required to implement common
network policies.}
\eat{Network policies (from above) often depend on shared information (for instance the identity of the user associated with a particular
IP address). The CAP theorem indicates that in general distributed data stores must make a choice between consistency and availability in the presence of
failures (partitions). We investigate the implications of the CAP theorem on policy enforcement in SDN networks and also look at the weakest consistency model
required by a variety of network policies.}

\cvlistitem {Data Driven Connectivity, built an online algorithm that can ensure perfect connectivity in the face of arbitrary link failures at line rate.}

\cvlistitem {SDNv2, an architecture for networks where the network edge consists of software switches running on general purpose computers. We investigate how
to structure the control and data plane for these networks and new applications enabled by this architecture}

\cvlistitem {SMIRC, a project that looked at using secure multiparty computation (SMPC) for computing BGP routing information. SMPC allows us to centralize
computation without requiring ASes to reveal private policy information, speeding up BGP convergence without requiring ASes to change their trust model.}

\cvlistitem {Structuring SDN for Large Networks, we look at how to structure the control plane for large (geographically and in terms of number of devices) SDN
networks.}

\cvlistitem {BlinkDB, a distributed database engine capable of running approximate queries. BlinkDB is currently deployed at Facebook and other companies.}

\cvlistitem {Symmetry Breaking for Constraint Programming, looked at methods by which one could speed up solving constraint satisfaction problems. These
problems are widely used (for instance to schedule UPS delivery trucks and NFL games) and are susceptible to an exponential growth in time, particularly when
many equivalent (symmetric) solutions exist. }

\cvlistitem {Hierarchical Bayesian Networks, looked at producing tools for recognizing hand drawn figures using hierarchical bayesian networks that were
themselves inspired by the structure of the human visual cortex.}

\section{Teaching Experience}
\cventry{08/2013-12/2013, 08/2012-12/2012}{TA for EE122 (Undergraduate Networking)}{UC Berkeley}{Berkeley, CA}{}{
\begin{itemize}
\item Taught about 20 students in a weakly discussion section.
\item Designed problem sets and projects, ran a project involving plug computers
\end{itemize}}
\cventry{01/2006--05/2008}{Undergraduate TA}{Brown University}{Providence, RI}{}{
\begin{itemize}
\item TAed CS51 (Models of Computation), CS138 (Distributed Computing), CS166 (Security), CS167 (Operating Systems) and CS169 (Operating Systems Lab).
\item Helped with designing new content and projects for distributed systems and security.
\item Helped design and grade problem sets and projects.
\end{itemize}
}
\cventry{08/2005--12/2005}{Reader}{Brown University}{Providence, RI}{}{
    \begin{itemize}
    \item Graded problem sets and exams for Honors Linear Algebra.
    \end{itemize}
}
\section{Industry Experience}
\cventry{2008--2011}{Software Developer}{Microsoft}{Redmond, WA}{}{\begin{itemize}
\item Worked on the kernel for Midori, an experimental operating system based on Singularity.
\item Designed, implemented and maintained Midori's performance counting and event logging infrastructure.
\item Designed and implemented timer infrastructure for Midori. Midori relied purely on core-local time and allowed applications to specify a tradeoff between
time accuracy and performance.
\item Ported Midori from x86-64 to NVidia's Tegra2 multicore ARM processor.
\end{itemize}}%
\cventry{06/2007--08/2007}{Software Engineering Intern}{Electronic Arts}{Redwood City, CA}{}{\begin{itemize}
\item Developed memory management tools for the PlayStation 3.
\end{itemize}}
\cventry{06/2006--08/2006}{Software Engineering Intern}{Bloomberg LP}{New York, NY}{}{\begin{itemize}
\item Developed network debugging tools for discovering the cause for latency spikes in data transfer from financial markets.
\end{itemize}}

\section{Publications}
\nociteconference{*}
\bibliographystyleconference{plainyr-rev}
\bibliographyconference{conference}                   % 'publications' is the name of a BibTeX file

\nociteworkshop{*}
\bibliographystyleworkshop{plainyr-rev}
\bibliographyworkshop{workshop}                   % 'publications' is the name of a BibTeX file


\nocitedemos{*}
\bibliographystyledemos{plainyr-rev}
\bibliographydemos{demos}                   % 'publications' is the name of a BibTeX file
\section{References}
Available on request
\end{document}


%% end of file `template.tex'.
