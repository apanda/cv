\documentclass[11pt,letterpaper,sans]{moderncv}        % possible options include font size ('10pt', '11pt' and '12pt'), paper size ('a4paper', 'letterpaper', 'a5paper', 'legalpaper', 'executivepaper' and 'landscape') and font family ('sans' and 'roman')
\usepackage{xspace}
\newcommand\eat[1]{}
%\newcommand{\ie}{{\it i.e.,}\xspace}

% moderncv themes
%\moderncvstyle{classic}                             % style options are 'casual' (default), 'classic', 'oldstyle' and 'banking'
\moderncvstyle{classic}
\moderncvcolor{black}                               % color options 'blue' (default), 'orange', 'green', 'red', 'purple', 'grey' and 'black'


\newcommand{\cvdoublecolumn}[2]{
    \begin{minipage}[t]{0.47\textwidth}#1\end{minipage}%
    \hfill%
    \begin{minipage}[t]{0.47\textwidth}#2\end{minipage}%
}


% usage: \cvreference{name}{address line 1}{address line 2}{address line 3}{address line 4}{e-mail address}{phone number}
% Everything but the name is optional
% If \addresssymbol, \emailsymbol or \phonesymbol are specified, they will be used.
% (Per default, \addresssymbol isn't specified, the other two are specified.)
% If you don't like the symbols, remove them from the following code, including the tilde ~ (space).

\newcommand{\cvreference}[7]{%
    \textbf{#1}\newline% Name
    \ifthenelse{\equal{#2}{}}{}{#2\newline}%
    \ifthenelse{\equal{#3}{}}{}{#3\newline}%
    \ifthenelse{\equal{#4}{}}{}{#4\newline}%
    \ifthenelse{\equal{#5}{}}{}{#5\newline}%
    \ifthenelse{\equal{#6}{}}{}{\texttt{#6}\newline}%
    \ifthenelse{\equal{#7}{}}{}{#7}}
%\renewcommand{\familydefault}{\sfdefault}         % to set the default font; use '\sfdefault' for the default sans serif font, '\rmdefault' for the default roman one, or any tex font name
\renewcommand*{\subsectionfont}{\large\mdseries\upshape}
\renewcommand*{\subsectionstyle}[1]{\subsectionfont\textbf{#1}}
%\nopagenumbers{}                                  % uncomment to suppress automatic page numbering for CVs longer than one page

% character encoding
%\usepackage[utf8]{inputenc}                       % if you are not using xelatex ou lualatex, replace by the encoding you are using
%\usepackage{CJKutf8}                              % if you need to use CJK to typeset your resume in Chinese, Japanese or Korean

% adjust the page margins
\usepackage[scale=0.75]{geometry}
%\setlength{\hintscolumnwidth}{3cm}                % if you want to change the width of the column with the dates
%\setlength{\makecvtitlenamewidth}{10cm}           % for the 'classic' style, if you want to force the width allocated to your name and avoid line breaks. be careful though, the length is normally calculated to avoid any overlap with your personal info; use this at your own typographical risks...

% personal data
\name{Aurojit}{Panda}
\title{Curriculum Vitae}                               % optional, remove / comment the line if not wanted
\address{60 Fifth Ave, Room 405}{New York, NY 10003}   % optional, remove / comment the line if not wanted; the "postcode city" and "country" arguments can be omitted or provided empty
\phone[mobile]{+1~(401)~323~1524}                   % optional, remove / comment the line if not wanted; the optional "type" of the phone can be "mobile" (default), "fixed" or "fax"
%\phone[fixed]{+2~(345)~678~901}
%\phone[fax]{+3~(456)~789~012}
\email{apanda@cs.nyu.edu}                               % optional, remove / comment the line if not wanted
\renewcommand*\httplink[2][]{{\urlstyle{sf}\expandafter\href#2}}
\homepage{{https://cs.nyu.edu/~apanda/}{cs.nyu.edu/{\textasciitilde}apanda}}                         % optional, remove / comment the line if not wanted
%\social[linkedin]{john.doe}                        % optional, remove / comment the line if not wanted
%\social[twitter]{apanda}                             % optional, remove / comment the line if not wanted
%\social[github]{apanda}                              % optional, remove / comment the line if not wanted
%\extrainfo{additional information}                 % optional, remove / comment the line if not wanted
%\photo[64pt][0.4pt]{picture}                       % optional, remove / comment the line if not wanted; '64pt' is the height the picture must be resized to, 0.4pt is the thickness of the frame around it (put it to 0pt for no frame) and 'picture' is the name of the picture file
%\quote{Some quote}                                 % optional, remove / comment the line if not wanted

% to show numerical labels in the bibliography (default is to show no labels); only useful if you make citations in your resume
%\makeatletter
%\renewcommand*{\bibliographyitemlabel}{\@biblabel{\arabic{enumiv}}}
\makeatother
%\renewcommand*{\bibliographyitemlabel}{[\arabic{enumiv}]}% CONSIDER REPLACING THE ABOVE BY THIS
\renewcommand*{\bibliographyitemlabel}{$\bullet$}% CONSIDER REPLACING THE ABOVE BY THIS

% bibliography with mutiple entries
\usepackage{multibib}
\newcites{conference,journals,workshop,demos,trs}{{Conferences},{Journals},{Workshops},{Demos},{Technical Reports}}



%----------------------------------------------------------------------------------
%            content
%----------------------------------------------------------------------------------
\begin{document}
%-----       resume       ---------------------------------------------------------
\makecvtitle
\section{Research Interests}
Computer Systems, Distributed Systems, Networking

\section{Education}
\cventry{2011--2017}{Ph.D. Computer Science}{University of California}{Berkeley, CA}{}{Advisor: Scott Shenker}  % arguments 3 to 6 can be left empty
\cventry{2004--2008}{Sc.B. Math--Computer Science}{Brown University}{Providence, RI}{}{Honors in Math--Computer Science\\
Advisor: Meinolf Sellmann}

\section{Professional Employment}
\cventry{Aug 2018--}{Assistant Professor}{Courant Institute, New York University}{New York, NY}{}{}
\cventry{2017--2018}{Researcher}{International Computer Science Institute}{Berkeley, CA}{}{}
\cventry{2017--2018}{Software Developer}{Nefeli Networks}{Berkeley, CA}{}{%
%Working on a NFV platform based on E2, NetBricks and Bess.
}
\cventry{2011--2017}{Research Assistant}{UC Berkeley}{Berkeley, CA}{}{}
\cventry{2008--2011}{Software Developer}{Microsoft}{Redmond, WA}{}{}%\begin{itemize}
%\item Worked on the kernel for Midori, an experimental operating system based on Singularity.
%\item Designed, implemented and maintained Midori's performance counting and event logging infrastructure.
%\item Designed and implemented timer infrastructure for Midori. Midori relied purely on core-local time and allowed applications to specify a tradeoff between
%time accuracy and performance.
%\item Ported Midori from x86-64 to NVidia's Tegra2 multicore ARM processor.
%\end{itemize}}%
\cventry{Summer '07}{Software Engineering Intern}{Electronic Arts}{Redwood City, CA}{}{%
%\begin{itemize}
%\item Developed memory management tools for the PlayStation 3.
%\end{itemize}
}
\cventry{Summer '06}{Software Engineering Intern}{Bloomberg LP}{New York, NY}{}{%
%\begin{itemize}
%\item Developed network debugging tools for inferring the cause for latency spikes in data transfer from financial markets.
%\end{itemize}}
}

% \section{Teaching Experience}
% \cventry{Spring '20}{Operating Systems}{NYU}{New York, NY}{}{}
% \cventry{Fall '19}{Computer Networks}{NYU}{New York, NY}{}{}
% \cventry{Fall '18}{Distributed Systems}{NYU}{New York, NY}{}{}
%\cventry{\hspace{2em}\ \ Fall '13\newline Fall '12}{TA for EE122 (Undergraduate Networking)}{UC Berkeley}{Berkeley, CA}{}{}
%\cventry{2006--2008}{Undergraduate TA}{CS Department, Brown University}{Providence, RI}{}{
%\begin{itemize}
%\item CS51 (Models of Computation), CS138 (Distributed Computing), CS166 (Security), CS167 (Operating Systems) and CS169 (Operating Systems Lab).
%\end{itemize}
%}
%\cventry{Fall '05}{Reader}{Math Department, Brown University}{Providence, RI}{}{
%}

\section{Awards}
\cvlistitem{NSF Career Award 2021}
\cvlistitem{Google Research Scholar Award 2021}
\cvlistitem{VMWare Early Career Faculty Grant 2018}
\cvlistitem{Demetri Angelakos Memorial Achievement Award, Berkeley EECS 2016-17}
\cvlistitem{Best Student Paper, SIGCOMM 2015}
\cvlistitem{Best Paper, EuroSys 2013}
\cvlistitem{Qualcomm Innovation Fellowship 2012}

\section{Grants}
%\cvlistitem{Microsoft Azure Research Award 2018 (\$5000 cloud credit)}
\cvlistitem{NSF CAREER: Assertions for Distributed Applications (2022-2027) (\$
700\,002)}
\cvlistitem{Google Research Award 2021 (\$ 80\,000 unrestricted gift)}
\cvlistitem{NSF EAGER: Towards an Extensible Internet (2020-2021) (\$ 30\,067)}
\cvlistitem{Microsoft Research Gift 2020 (\$ 80\,000 unrestricted gift, joint with Jinyang Li)}
\cvlistitem{NSF PPoSS: Planning: Making Smart Use of SmartNICs (2020-2021) (\$ 80\,000}
\cvlistitem{Microsoft Research Gift 2019 (\$ 80\,000 unrestricted gift, joint with Jinyang Li)}
\cvlistitem{Nefeli Networks Gift 2019 (\$ 50\,000 unrestricted grant)}
\cvlistitem{VMWare Early Career Faculty Award 2018 (\$ 35\,000 unrestricted grant)}

\section{Students Advised}
\cvlistitem{John Westhoff. PhD student (Fall 2018 -- Fall 2021)}
\cvlistitem{Changgeng Zhao. PhD student (Fall 2018 -- Fall 2021)}
\cvlistitem{Yu Cao. PhD student co-advised with Jinyang Li (Fall 2019 -- present)}
\cvlistitem{Jinkun Lin. PhD student co-advised with Jinyang Li (Fall 2019 -- present)}
\cvlistitem{Anqi Zhang. PhD student co-advised with Jinyang Li (Fall 2019 -- present)}
\cvlistitem{Vinayak Agarwal. Masters student (Summer 2021 -- present)}
\cvlistitem{Vivian Fang. Undergraduate student (Fall 2017 -- Fall 2018)}
\cvlistitem{Christopher Branner-Augmon. Undergraduate student (Fall 2019 -- Fall
2020)}



\section{Publications}
\nociteconference{*}
\bibliographystyleconference{plainyr-rev}
\bibliographyconference{conference}                   % 'publications' is the name of a BibTeX file

\nocitejournals{*}
\bibliographystylejournals{plainyr-rev}
\bibliographyjournals{journals}                   % 'publications' is the name of a BibTeX file

\nociteworkshop{*}
\bibliographystyleworkshop{plainyr-rev}
\bibliographyworkshop{workshop}                   % 'publications' is the name of a BibTeX file


\nocitedemos{*}
\bibliographystyledemos{plainyr-rev}
\bibliographydemos{demos}                   % 'publications' is the name of a BibTeX file

\nocitetrs{*}
\bibliographystyletrs{plainyr-rev}
\bibliographytrs{tr}                   % 'publications' is the name of a BibTeX file

%\section{Research}
%%\cvlistitem{\textbf{NetBricks}}: 
%%\cvlistitem {\textbf{ZCSI}: Network operators are increasingly moving from hardware middleboxes to software network
%%functions. The move to software reduces vendor lock-in, makes it easier to deploy new features, and provides benefits
%%for consolidation. The current model for deploying these network functions envisions using virtual machines (or
%%containers) connected using virtual switches. While this model has served workloads in the cloud well, NFs need to
%%process millions of packets per-second while adding no more than a few microseconds of delay in order to meet SLAs. VMs
%%and containers add high-overheads for this. ZCSI is a new NF framework which leverages software isolation to provide the
%%same guarantees as VMs without incurring this overhead. Furthermore, ZCSI allows NFs to be written in high-level, safe
%%languages, enhancing developer productivity and allowing faster development. ZCSI achieves this while improving
%%performance when compared to today's deployments.}

%%\cvlistitem {\textbf{NetBricks}: The use of ZCSI requires that NFs be rewritten. Writing an NF today requires that NF
%%writers spend a great deal of time implementing and optimizing many basic building blocks (e.g., algorithms that split
%%packets across cores, cache-aware data structures to reduce cross-core contention, etc.). However these building blocks
%%are not easily generalized, the optimizations within these blocks is dependent on the algorithm being implemented, and
%%design choices for other building blocks. NetBricks is a new framework for NFV where NFs are specified using a data flow
%%graph whose nodes are drawn from a fixed set provided by the framework. This enables NetBricks to perform automatic
%%reasoning and optimization, freeing the NF developer from these concerns.}

%\cvlistitem {\textbf{BESS}: The Berkeley Extensible Software Switch is a virtual switch that has been specifically
%optimized for NFV deployments. It is written to allow easy extensibility (allowing custom modules to be run), and offers
%significant performance improvements over existing solutions like OpenVSwitch.}

%\cvlistitem {\textbf{E2}: The move to NFV makes it essential that we build and provide tools for managing how NFs are
%run and scaled in a cluster. E2 is a system that manages this, it builds on top of BESS (which is used to detect cases
%where an individual NF is overloaded, and scale the NF as appropriate). E2 also provides support for NF fault tolerance
%(through the use of FTMB), and allows operators to enforce QoS requirements, etc.}

%\cvlistitem {\textbf{Model Checking for Networks}: Networks include middleboxes and other stateful elements whose
    %behavior is dictated by previously encountered packets. We developed a tool to verify isolation invariants (can
    %packets with certain characteristics be delivered to certain end hosts, and can data originating at a particular
    %host end up somewhere). We also developed theoretical foundations showing that if one uses middlebox models with
    %certain simple limitations and runs checks in loop free network then verification is guaranteed to terminate.}

%\cvlistitem {\textbf{Data-Aware Cluster Scheduling for Sampling Jobs}: An increasing portion of cluster computing jobs
%use sampled data, either to reduce computational time using approximate querying or when computing stochastic machine
%learning functions. In this work, we looked at how cluster schedulers could use sampling information to significantly
%improve the performance of cluster computing jobs.}

%\cvlistitem {\textbf{Network Consistency Models}: Networks increasingly enforce policy, which ultimately requires that
%several of the middleboxes/routers in a network have a consistent view of the network. However, the CAP theorem says
%that this must necessarily come at the cost of giving up availability in the presence of partitions. In this work we
%analyzed the effects of this trade-off.}

%\cvlistitem {\textbf{Data Driven Connectivity}: Minimizing the network disruption due to link or router/switch failures
    %continues to be an important concern.
    %We developed an online algorithm that provides perfect forwarding
%resilience \textit{i.e.,} guarantees that in an uncongested, connected network no set of link failures results in a packet drop. We
%have also had subsequent work on theory that shows the limits of resilience that can be achieved using just static
%routing tables with no updates.}

%\cvlistitem {\textbf{SDNv2}: The rise of NFV and other similar technologies has meant that general purpose computers are
%increasingly available at the network edge. We designed a new internet architecture that allowed ISPs to leverage these
%computers to provide increased flexibility to the customers.}

%\cvlistitem {\textbf{SMIRC}: We asked the question, can secure multiparty computation (SMPC) can be used to compute BGP
%routing information. The use of SMPC would allows BGP computation to be centralized without requiring ASes to reveal
%private policy information, speeding up BGP convergence without requiring ASes to change their trust model.}

%\cvlistitem {\textbf{Recursive SDN}: We looked at how one could scale SDN controllers so they can be used to control ISP
    %networks which tend to both be numerically larger, and geographically wide-spread. Our solution is to use hierarchy
    %and recursive algorithms.}


    %\cvlistitem {\textbf{BlinkDB}: We built a fast, approximate database on top of Spark. BlinkDB used sampling and
    %innovative statistical techniques to provide approximate answers and quality measurements for arbitrary SQL query.
%BlinkDB has since been deployed at companies like Facebook, is commercially supported by other companies.}

%\cvlistitem {\textbf{Symmetry Breaking for Constraint Programming}: We looked at methods by which one could speed up
%solving constraint satisfaction problems. These problems are widely used (for instance to schedule UPS delivery trucks
%and NFL games) and are susceptible to an exponential growth in time, particularly when many equivalent (symmetric)
%solutions exist. }

\section{Invited Talks}
    \textbf{Programming the Edge}
    \begin{itemize}
        \item Hebrew University Summer School on Networking. June 2019.
        \item Akraino Edge Summit. San Diego. August 2019.
    \end{itemize}
    \textbf{A New Approach to Network Function Virtualization}
    \begin{itemize}
        \item USC. February 2017.
        \item NYU. February 2017.
        \item University of Wisconsin. February 2017.
        \item University of Chicago. March 2017.
        \item MPI SWS. March 2017.
        \item EPFL. March 2017.
        \item UT Austin. April 2017.
        \item Microsoft Research. April 2017.
        \item IETF NFV Research Group. September 2017.
    \end{itemize}
    \textbf{NetBricks: Taking the V out of NFV}
    \begin{itemize}
        \item Intel Research. October 2016
        \item Google Platforms and Networking. October 2016
    \end{itemize}
    \textbf{VMN: Verifying Networks with Mutable Datapaths}
    \begin{itemize}
        \item Invited speaker at NetPL. August, 2016.
        \item Dagstuhl - Formal Foundations for Networking. February 2015.
        \item Bellairs Seminar on Network Verification. February 2020.
    \end{itemize}
\section{Service}
    \begin{itemize}
        \item Reviewer for:
            \begin{itemize}
                \item OSDI 2021
                \item SIGCOMM (2019, 2020)
                \item NSDI (2019, 2020, 2022)
                \item EuroSys (2019)
                \item USENIX ATC (2019, 2020)
                \item CoNext (2018)
                \item HotNets (2018, 2021)
                \item HotCloud (2020)
                \item SIGCOMM CCR (2017, 2018, 2020)
                \item ACM/IEEE Transactions on Networking (2017)
                \item SOSR (2018, 2020)
                \item MobiSys (ERC 2018)
                \item ANCS (2018)
                \item EuroSys Doctoral Workshop (2018)
                \item KBNets (2018)
                \item SecSoN (2018)
                \item ACM/IEEE Transactions on Networking (2018)
                \item EuroP4 (2019, 2020)
            \end{itemize}
        \item Travel Grants chair for ANCS 2018.
        \item Preview Sessions Chair for NSDI 2019.
        \item Publication chair for SIGCOMM 2020. 
        \item Student Research Competition chair for SOSP 2021. 
    \end{itemize}
\section{References}
Available on request
\eat{
\cvdoublecolumn{\cvreference{Prof. Scott Shenker}
{Electrical Engineering and Computer Science}
{University of California, Berkeley}
{}
{}
{shenker@icsi.berkeley.edu}
{}
}
{\cvreference{Prof. Sylvia Ratnasamy}
{Electrical Engineering and Computer Science}
{University of California, Berkeley}
{}
{}
{sylvia@eecs.berkeley.edu}
{}
}
\cvdoublecolumn{\cvreference{Prof. Ion Stoica}
{Electrical Engineering and Computer Science}
{University of California, Berkeley}
{}
{}
{istoica@cs.berkeley.edu}
{}
}
{\cvreference{Prof. Mooly Sagiv}
{School of Computer Science}
{Tel Aviv University}
{}
{}
{msagiv@acm.org}
{}
}

{\cvreference{Prof. Katerina Argyraki}
{School of Computer and Communication Sciences}
{Ecole Polytechnique Federale de Lausanne}
{}
{}
{katerina.argyraki@epfl.ch}
{}
}
}
\end{document}
